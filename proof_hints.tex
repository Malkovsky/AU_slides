\documentclass[10pt,a4paper]{article}
\usepackage[utf8]{inputenc}
\usepackage[russian]{babel}
\usepackage[OT1]{fontenc}
\usepackage{amsmath}
\usepackage{amsfonts}
\usepackage{amssymb}
\usepackage{amsthm}
\usepackage{hyperref}

\newcounter{thm}
\newcounter{lm}
\newcounter{def}


\newtheorem{theorem_ru}{Теорема}[section]
\newtheorem{lemma_ru}{Лемма}[section]
\newtheorem{definition_ru}{Определение}[def]

\newenvironment{sketch}
    {\begin{proof}[Идея доказательства]
    }
    { 
    \end{proof}
    }


\DeclareMathOperator{\argmin}{argmin}
\DeclareMathOperator{\interior}{Int}
\DeclareMathOperator{\conv}{Conv}
\DeclareMathOperator*\uplim{\overline{lim}}




\title{Proof hints}
\begin{document}
\begin{center}
\Large{\textbf{Proof hints}}
\end{center}
\tableofcontents

\section{Алгебра}
\subsection{Ортогонализация Грама-Шмидта}
Ортогональность и идентичность линейных оболочек проверяется непосредственно по определению, достаточно использовать индукцию и определение
\[
y_k=x_k-\sum_{i=1}^{k-1}\frac{\langle x_k, y_i\rangle}{\langle y_i, y_i\rangle}
\]
\subsection{Линейные функции}
\begin{theorem_ru}
Пусть $f:K^n\rightarrow K^m$ -- линейная функция, тогда существует $A\in K^{m\times n}$ такая, что 
$$
f(x)=Ax
$$
\end{theorem_ru}
\begin{proof}[Идея доказательства]
Взять стандартный базис $e_n^1, \ldots, e_n^n$ в $K^n$ и рассмотреть матрицу $A=(a_{ij}), a_{ij}=f_j(e_n^i)$, из линейности $f(x)=Ax$.
\end{proof}
\begin{theorem_ru}
Любое конечномерное пространство $V$ размерностью $n$ над $K$ изоморфно $K^n$
\end{theorem_ru}
\begin{sketch}
Взять произвольный базис $v_1, \ldots, v_n$ в $V$ и рассмотреть линейную функцию, которая отображает стандартный базис в $K^n$ в рассмотренный базис в $V$.
\end{sketch}

\subsection{Свойства вещественных симметричных матриц}
\begin{enumerate}
\item \textit{Все собственные числа положительны}:\\
Для ненулевого собственного вектора $v:~Av=\lambda v$ используя симметричность $A$ и $\overline{\lambda}\overline{v}=\overline{A}\overline{v}=A\overline{v}$ показать, что $\lambda \overline{v}^Tv=\overline{\lambda} \overline{v}^Tv$
\item \textit{Диагонализируемость}:\\
Рассмотреть некоторый собственный нормированный вектор $v$, дополнить его до ортонормированного базиса $v, y_1, \ldots, y_n$, если матрица $B$ составлена из столбцов $v, y_1, \ldots, y_n$, то $B^T=B^{-1}$, а матрица $B^TAB$ имеет вид
$\left[
\begin{array}{cc}
\lambda & 0\\
0 & C
\end{array}
\right]
$, где $C$ -- некоторая симметричная матрица размера $n-1$. Далее индукция или анализ характеристического многочлена.
\end{enumerate}

\section{Элементы выпуклого анализа}
\subsection{Критерии выпуклости}
Следующие определения эквивалентны:
\begin{enumerate}
\item $f$ сильно выпукла с константой $m$.
\item 
$$
f(y)\geq f(x)+\nabla f(x)^T(y-x)+\frac{m}{2}||y-x||^2
$$
\item 
$$
(\nabla f(y) - \nabla f(x))^T(y-x)\geq m||y-x||^2
$$
\item 
$$
\nabla^2f(x)\succeq mI
$$
\end{enumerate}
\begin{sketch}
\begin{itemize}
\item[$1\Leftrightarrow 2$] $\Rightarrow$: перегруппировать и устремить $t$ к нулю  учитывая
$$
\lim_{t\rightarrow 0}\frac{f(x+t(y-x))-f(x)}{t}=\nabla f(x)^T(y-x)
$$ 
$\Leftarrow$: применить неравенство для пар $(x+t(y-x), x)$, $(x+t(y-x), y)$, сложить эти неравенства с коэффициентами $t, 1-t$.
\item[$2\Leftrightarrow 3$] $\Rightarrow$: сложить два неравенства для пар $(x, y)$ и $(y, x)$.\\
$\Leftarrow$: применить формулу Ньютона-Лейбница
$$
f(y)=f(x)+\int_0^1\nabla f(x+t(y-x))^T(y-x)dt
$$ 
\item[$3\Rightarrow 4$] Применить $2$ к паре $x, x+t(y-x)$, поделить на $t^2$ и устремить $t$ к нулю.
\item[$4\Rightarrow 2$] Применить формулу Тейлора с остатком в интегральной форме
$$
f(y)=f(x)+\nabla f(x)^T(y-x) + \frac{1}{2}\int_0^1(y-x)^T\nabla^2f(x+t(y-x))(y-x)dt
$$
\end{itemize}
\end{sketch}

\textit{Замечание.} Аналогичные утверждения получаются для функции с липшицевым градиентом, доказательства аналогичны.

\subsection{Три теоремы о выпуклых множествах}
\begin{theorem_ru}[О надграфике выпуклой функции]
Пусть $\mathcal{D}$ -- выпуклое множество.
$f:\mathcal{D}\subset \mathbb{R}^n\rightarrow \mathbb{R}$ -- выпуклая функция тогда и только тогда, когда множество $\mathbb{G}=\{(x, y)~|~x\in \mathcal{D}, y\in \mathbb{R}, y\geq f(x)\}$ выпукло.
\end{theorem_ru}
\begin{sketch}
Прямолинейное применение определения.
\end{sketch}
\begin{theorem_ru}[О выпуклости множества, заданном неравенством]
Если $f:\mathcal{D}\subset \mathbb{R}^n\rightarrow \mathbb{R}$ -- выпуклая функция, $\mathcal{D}$ -- выпуклое множество, $c\in \mathbb{R}$, тогда множество $\mathbb{E}=\{x\in \mathcal{D}~|~f(x)\leq c\}$ выпукло.
\end{theorem_ru}
\begin{sketch}
Прямолинейное применение определения.
\end{sketch}
\begin{theorem_ru}[О представлении замкнутых выпуклых множеств]
Пусть $\mathcal{D}\in \mathbb{R}^n$ -- замкнутое выпуклое множество, тогда 
$$
\mathcal{D}=\cap_{x\in \mathbb{R}^n}\{y\in \mathbb{R}^n~|~\langle x, y\rangle\leq \phi_\mathcal{D}(x)\}
$$
\end{theorem_ru}
\begin{sketch}
Если $x\in \mathcal{D}$, то из определения $\phi_\mathcal{D}$ следует, что $x$ принадлежит правой части. Если $x\in \mathcal{D}$, то существует гиперплоскость, отделяющая $x$ от $\mathcal{D}$, достаточно посмотреть на $\phi_\mathcal{D}(a)$, где $a$ -- нормаль разделяющей гиперплоскости.
\end{sketch}

\section{Сходимость рекуррентных процессов}
\subsection{Теорема Банаха}
\begin{theorem_ru}[Банаха о сжимающем отображении]
Пусть $(X, \delta)$ -- непустое полное метрическое пространство (банахово пространство), $f$ -- сжимающее отображение на $X$ с константой $q$, тогда 
\begin{itemize}
\item Существует единственная неподвижная точка $x^*\in X$ функции на $f$ на $X$.
\item При любом $x_0\in X$ рекуррентная последовательность $x_{k+1}=f(x_k)$ сходится к $x^*$.
\end{itemize} 
\end{theorem_ru}

\begin{sketch}
Доказать, что последовательность является фундаментальной, используя неравенство треугольника и факт, что $\delta(x_k, x_{k+1})$ сходится не медленней, чем геометрическая. Единственность доказывается от противного.	
\end{sketch}

\subsection{Сходимость линейного процесса}
\begin{theorem_ru}
Последовательность, заданная рекуррентным соотношением $x_{k+1}=Ax_k+b$ сходится при $\sigma(A)<1$.
\end{theorem_ru}
\begin{sketch}
Используя жорданову форму доказать, что $\forall \epsilon>0~\exists C>0:~\forall k ||A||^k\leq C(\sigma(A)+\epsilon)^k$, дальше схема доказательства такая же, как и для теоремы Банаха.
\end{sketch}

\section{Множители Лагранжа и Условия ККТ}
\subsection{Условия стационарности для задач на произвольных множествах}
\begin{equation}\label{general_formulation}
\begin{array}{ll}
\mbox{минимизировать } & f(x),\\
\mbox{при условии } & x\in \mathcal{D}.
\end{array}
\end{equation}

\begin{theorem_ru}
Пусть в задаче \eqref{general_formulation} множество $\mathcal{D}$ выпукло, а функция $f$ дифференцируема в точке $x^*$, тогда для точки минимума $x^*$ выполняется условие
$$
\nabla f(x^*)^T(x-x^*)\geq 0,~~\forall x\in \mathcal{D}.
$$
\end{theorem_ru}
\begin{sketch}
От противного используя выпуклость $\mathcal{D}$ найти направление, для которого неравенство не выполняется и немного сдвинуться вдоль него.
\end{sketch}

\subsection{Метод Лагранжа}

\begin{equation}\label{cond_extr}
\begin{array}{ll}
\mbox{минимизировать} &f(x),\\
\mbox{при условии}    &g(x)=0_m.
\end{array}
\end{equation}

\begin{theorem_ru}
$x^*$ -- точка минимума задачи \eqref{cond_extr} и векторы $\nabla g_i (x^*)$ линейно независимы, тогда существует такой вектор $\lambda=(\lambda_1, \lambda_2, \ldots, \lambda_m)^T$, что
$$
\nabla f(x^*)+ \lambda^T\nabla g(x^*)=0_n.
$$
\end{theorem_ru}

\begin{sketch}
Использовав следствие из теоремы о неявном отображении доказать, что для любого $y$ одновременно ортогональному всем $\nabla g_i(x^*)$, $\nabla f(x^*)$ также ортогонален $y$.
\end{sketch}

\subsection{Условия ККТ}
\begin{lemma_ru}
Рассмотрим задачу
$$
\begin{array}{ll}
\mbox{минимизировать} & f(x) \\
 & h(x)=d,
\end{array}
$$
где $d\in \mathbb{R}^k$. Пусть $x^*(d), \lambda(d)$ -- решение задачи и соответствующие ему множители Лагранжа, удовлетворяющие
$$
\nabla f(x^*(d))+\lambda(d)^T \nabla h(x^*(d))=0_n,
$$
при этом система
\begin{align*}
\nabla f(x)+\lambda^T \nabla h(x)&=0_n\\
h(x)&=d
\end{align*}
регулярна относительно $x$ в точке $x^*(d)$, тогда
$$
\nabla_d f(x^*(d)) = -\lambda(d).
$$
\end{lemma_ru}
\begin{sketch}
Несколько использований правила производной композиции, а также факта, что $\nabla_d(x^*(d))=I_k$.
\end{sketch}

\begin{theorem_ru}
Для точки минимума $x^*$ задачи
$$
\begin{array}{ll}
\mbox{минимизировать } & f(x),\\
\mbox{при условии }    & g(x)\leq 0_m,\\
					   & h(x)=0_k.
\end{array}
$$
Выполняются условия
$$
\begin{array}{ll}
\mbox{1. Стационарность: }& \nabla f(x^*) + \lambda^T\nabla g(x^*)+\mu^T \nabla h(x^*)=0_n.\\
\mbox{2. Прямая выполнимость: }&g(x^*)\leq 0_m,~h(x^*)=0_k.\\
\mbox{3. Двойственная выполнимость: }&\lambda\geq 0_m~(\leq 0_m \mbox{ для задачи максимизации}).\\
\mbox{4. Дополняющая нежесткость: }& \lambda_i g_i(x^*)=0,~1\leq i\leq m.
\end{array}
$$
\end{theorem_ru}
\begin{sketch}
Свести к задаче с ограничениями в виде равенств добавлением переменных используя простой факт $g_i(x)\leq 0\Leftrightarrow \exists y_i:~g_i(x)+y_i^2=0$. Доказать $2$ используя лемму о чувствительности.
\end{sketch}

\section{Анализ градиентного спуска}
\begin{theorem_ru}[Постоянный шаг]
Пусть $f$ выпукла и дифференцируема на $\mathcal{D}$, градиент $f$ липшицев с константой $M>0$ на $S_f(x_0)$, $f$ ограничена снизу,
существует хотя бы одна точка минимума $x^*$, $\alpha_k=\alpha\in [0, 1/M]$,
тогда для последовательности $x_k$, генерируемой градиентным спуском, $f(x_k)$ убывает и, более того
$$
f(x_k)-f(x^*)\leq \frac{1}{2\alpha k}||x_0-x^*||^2.
$$
\end{theorem_ru}
\begin{sketch}
\begin{enumerate}
\item Используя $\alpha \leq 2 / M$ и липшицевость $\nabla f$ показать, что последовательность $f(x_k)$ убывает, заодно вывести неравенство
$$
f(x)-f(x^*)\geq \frac{1}{2M}||\nabla f(x)||^2
$$
\item Используя $\alpha \leq 1 / M$, выпуклость $f$ получить
$$
f(x_{k+1})-f(x^*)\leq \frac{1}{2\alpha}(||x_k-x^*||^2-||x_{k+1}-x^*||^2).
$$
\item Просуммировать неравенства из предыдущего пункта и учесть убывание $f(x_k)$.
\end{enumerate}
\end{sketch}

\textit{Замечание.} Анализ для \textit{backtracking line search} идентичен, различие заключается в том, что вместо постоянного $\alpha$ используется $\alpha_k\geq \min{1, \beta / M}$.

\begin{theorem_ru}[Постоянный шаг, сильная выпуклость] Пусть $f$ дифференцируема на $\mathcal{D}$, $\alpha_k\equiv \alpha\in (0, 2/M)$, $f$ сильно выпукла с константой $m>0$ на $S_f(x_0)$, градиент $f$ липшицев с константой $M\geq m$ на $S_f(x_0)$, тогда для последовательности $x_k$, генерируемой градиентным спуском,
$x_k$ cходится к единственной точке минимума $x^*$ $f$ на $\mathcal{D}$, $f(x_k)$ убывает и сходится к $f(x^*)$, более того для $q=1-2m\alpha+mM\alpha^2$
$$
f(x_k)-f(x^*)\leq q^k(f(x_0)-f(x^*)).
$$
\end{theorem_ru}

\begin{sketch}
Три неравенства:
\begin{enumerate}
\item Сильная выпуклость $f$.
$$
f(x)-f(x^*)\leq \frac{1}{2m}||\nabla f(x)||^2
$$
\item Сильная выпуклость $f$
$$
||x-x^*||\leq \frac{2}{m}||\nabla f(x)||.
$$
\item Определение градиентного спуска и липишицевость градиента 
$$
f(x_{k+1})\leq f(x_k)-\alpha\left(1-\frac{\alpha M}{2}\right)||\nabla f(x_k)||^2.
$$
\end{enumerate}
$1$+суммирование по $k$ в $3$ дают сходимость. $1+2$ дают скорость сходимости.
\end{sketch}

\begin{lemma_ru}[О $m$-сильно выпуклой $M$-гладкой функции]
Пусть $f:\mathcal{D}\rightarrow \mathbb{R}$ -- сильно выпуклая c параметром $m$ функция, $\nabla f$ удовлетворяет условию
Липшица с параметром $M$, т.~е.
$$
m||y-x||^2\leq (\nabla f(y)-\nabla f(x))^T(y-x)\leq M||y-x||^2,
$$
тогда для $f$ выполняется
$$
(\nabla f(y)-\nabla f(x))^T(y-x)\geq \frac{mM}{m+M}||y-x||^2+\frac{1}{m+M}||\nabla f(y)-\nabla f(x)||^2
$$
\end{lemma_ru}
\begin{sketch}
Два этапа
\begin{enumerate}
\item Выпуклая функция $g$ с $M$-липшицевым градиентом удовлетворяет неравенству
$$
g(y)\geq g(x)+\nabla g(x)^T(y-x)+\frac{1}{2(M-m)}||\nabla g(y)-\nabla g(x)||^2
$$
которое получается применением $f(x)-f(x^*)\geq \frac{1}{2M}||\nabla f(x)||^2$ к функции $\phi(y)=g(y)-\nabla g(x)^Ty$ ($\phi$ выпукла достигает минимума в точке $x$).
\item Функция $g(x)=f(x)-\frac{m}{2}x^2$ выпукла, а её градиент $M-m$-липшицев. Применяем к этой функции неравенство из предыдущего пункта и получаем утверждение леммы.
\end{enumerate}
\end{sketch}

\begin{theorem_ru}
Пусть $f$ дифференцируема на $\mathcal{D}$, $\alpha_k\equiv \alpha\in (0, 2/(M+m))$, $f$ сильно выпукла с константой $m>0$ на 
$\bar{B}(x^*, ||x_0-x^*||)$, градиент $f$ липшицев с константой $M\geq m$ на $\bar{B}(x^*, ||x_0-x^*||)$, тогда для последовательности $x_k$, генерируемой по правилу градиентным спуском,
$x_k$ cходится к единственной точке минимума $x^*$ $f$ на $\mathcal{D}$, более того для 
$$
||x_k-x^*||^2\leq \left(1-\frac{2\alpha mM}{M+m}\right)^k||x_0-x^*||^2
$$
\end{theorem_ru}

\begin{sketch}
Раскрыть $||x_{k+1}-x^*||^2=||(x_k-x^*)-\alpha\nabla f(x_k)||^2$ как квадрат суммы, применить неравенство из леммы и использовать $\alpha<2/(m+M)$.
\end{sketch}

\section{Решение линейных систем, минимизация квадратичных функций}
\subsection{Разложение Холеского}
\begin{theorem_ru}
Для квадратной матрицы $A$ разложение Холеского существует, если она симметрична и положительно определена, и не существует, если не симметрична или не положительно полуопределена.
\end{theorem_ru}
\begin{sketch}
Использовать существование $LU$ разложения для получения $LDL^T$ разложения.
\end{sketch}

\subsection{Итеративные методы}
\begin{theorem_ru}
Если матрица $I-DA$ положительно определена, $\sigma(I-DA)<1$ и $D$ обратима, то последовательность 
$$
x_{k+1}=(I-DA)x_k+Db
$$
сходится к сходится решению системы $Ax=b$.
\end{theorem_ru}
\begin{sketch}
Прямое следствие теоремы о сходимости линейных процессов.
\end{sketch}

\begin{theorem_ru}[Гершгорина о кругах]
Пусть $A=(a_{ij})\in \mathbb{C}^{n\times n}$. Для любого собственного числа $\lambda$ матрицы $A$ найдется $1\leq i\leq n$ такое, что
$$
|\lambda-a_{ii}|\leq \sum_{j\neq i}|a_{ij}|
$$
\end{theorem_ru}
\begin{sketch}
В равенстве $Ax=\lambda x$ посмотреть на наибольшую по модулю компоненту $x$ и соответствующее ей равенство $\sum_{j=1}^na_{ij}x_j=\lambda x_i$.
\end{sketch}

\subsection{Метод сопряженных направлений}
\textbf{Сходимость за конечное число итераций}:\\
Используя сопряженность $d_i$ доказать, что имеет место равенство
$$
x_0-x^*=\sum_{i=0}^{k-1}\alpha_id_i,
$$
где $\alpha_i$ -- используемый размер шага (минимум по направлению) на итерации $i$.\\
\vspace{1em}
\textbf{Сложность} $\mathcal{O}(n^2m)$:\\
Использование процедуры Грама-Шмидта с эффективным умножением матрицы на вектор.

\subsection{Метод сопряженных градиентов}
\textbf{Сложность} $\mathcal{O}(nm)$:\\
\begin{enumerate}
\item Если использовать градиенты в качестве $v_k$, то метод генерирует последовательность Крылова для $A, Ax_0-b$.
\item Условия оптимальности (условия Лагранжа) дают ортогональность градиентов
\item Применив рекуррентное соотношение для $x_k$ и ортогональность $v_k$ получается формула для пересчета $d_k$, использующая только фиксированное число умножений матрица-вектор, вектор-вектор.
\end{enumerate}

\subsection{Асимптотический анализ}
Показать, что метод сопряженных градиентов обладает свойством
$$
x_k-x^*=P_k(A)(x_0-x^*)
$$
где $P_k$ -- полином степени $k$ с младшим коэффициентом $1$ такой, что $A$-норма левой части минимальна. Если взять конкретный полином степени $k$ c младшим коэффициентом $1$, то он даст верхнюю оценку на сходимость метода. Оптимальным с точки зрения асимптотического анализа является полином Чебышёва.

\section{Оптимальные методы}

\subsection{Тяжелый шарик}
Метод тяжелого шарика имеет вид
$$
x_{k+1} = x_k - \alpha(Ax_k-b)+\beta(x_k-x_{k-1})
$$
и может быть переписан в виде
$$
y_{k+1}=By_k+c,
$$
где $y_k=\left[\begin{array}{c}x_k\\ x_{k-1}\end{array}\right]$. Из симметричности $A$ матрицу $B$ можно представить в блочной диагональной матрицы с блоками $2\times 2$ похожей структуры. Собственные числа матрицы $2\times 2$ легко вычисляются аналитически, оказывается эффективно сделать числа положительными (правильно подобрать $\beta$).

\subsection{Метод Чебышёва}
Метод имеет вид
$$
x_{k+1} = x_k - \alpha_k(Ax_k-b)+\beta_k(x_k-x_{k-1}),
$$
при этом $\alpha_k, \beta_k$ подобраны так, что $x_{k}-x^*=P_k(A)(x_0-x^*)$, где $P_k$ -- соответствующий многочлен Чебышёва, построенный на отрезке границ спектра $A$. Вывод способа подсчета коэффициентов можно условно разделить на два этапа:
\begin{enumerate}
\item Нахождение рекуррентного соотношения второго порядка для $P_k$ используя рекуррентное соотношение для стандартного полинома Чебышёва.
\item $\alpha_k, \beta_k$ можно вывести непосредственной подстановкой равенства $x_{k}-x^*=P_k(A)(x_0-x^*)$ в рекуррентное соотношение для $x_k$ учитывая $Ax_k-b=A(x_k-x^*)$.
\end{enumerate}

\subsection{Метод Нестерова}
Рекуррентные соотношения для оценивающих последовательностей:
\begin{enumerate}
\item Чтобы показать общий вид $\phi_k(x)=\phi_k^*+\frac{\gamma_k}{2}||x-v_k||^2$ достаточно просто дважды продифференцировать рекуррентное соотношение. Учитывая $\nabla^2\phi_k(x)=\gamma_kI$ из дифференцирования также легко вывести рекуррентное соотношение для $\gamma$.
\item Рекуррентное соотношение для $v$ выводится из условия $\nabla \phi_k(v_k)=0$.
\item Рекуррентное соотношение для $\phi^*$ можно вывести подставив $y_k$ в рекуррентное соотношение для $\phi$ и воспользоваться посчитанным $v_{k+1}$.
\end{enumerate}
После введения таких оценивающих последовательностей нужно научиться строить последовательность $x_k$ так, чтобы выполнялось $f(x_k)\leq \phi^*_k$, для этого предлагается воспользоваться индукцией и правильным выбором $\alpha_k, y_k$. После использования выпуклости получается неравенство
\begin{align*}
\phi_{k+1}^*&\geq(1-\alpha_k)f(x_k)+\alpha_kf(y_k)-\frac{\alpha_k^2}{2\gamma_{k+1}}||\nabla f(y_k)||^2\\
&~~~+\frac{\alpha_k(1-\alpha_k)\gamma_k}{\gamma_{k+1}}
\left(\frac{m}{2}||y_k-v_k||^2+\nabla f(y_k)^T(v_k-y_k)\right).
\end{align*}

\begin{enumerate}
\item $y_k$ можно выбрать так, чтобы обнулить выражение в последних скобках.
\item $\alpha_{k}$ можно выбрать так, чтобы выполнялось $\frac{\alpha_k^2}{\gamma_{k+1}}=\frac{1}{M}$, что, учитывая рекуррентное соотношение для $\gamma$,  дает квадратное уравнение на $\alpha_k$.
\item Наконец осталось подобрать $x_k$ так, чтобы выполнялось $f(x_{k+1})\leq f(y_k)-\frac{1}{2M}||\nabla f(y_k)||^2$, что можно получить сделав шаг обычного градиентного спуска из $y_k$.
\end{enumerate}

\begin{lemma_ru}
Если в общей схеме взять $\gamma_0\geq m$, то 
$$
\lambda_k=\prod_{i\leq k-1}(1-\alpha_k)\leq \min\left\{\left(1-\sqrt{\frac{m}{M}}\right)^k, \frac{4M}{(2\sqrt{M}+k\sqrt{\gamma_0})^2}\right\}
$$
\end{lemma_ru}

\begin{sketch}
Первая часть: по индукции доказать, что $\alpha_k\geq \sqrt{\frac{m}{M}}$. \\
Вторая часть: доказать по индукции $\gamma_{k+1}\geq \gamma_0\lambda_{k+1}$, рассмотреть $\omega_k=1/\sqrt{\lambda_k}$, привести к одному знаменателю $\omega_{k+1}-\omega_k$, домножить на $\sqrt{\lambda_k}+\sqrt{\lambda_{k+1}}$, воспользоваться убыванием $\lambda_k$, рекуррентными соотношениями и доказанным ранее.
\end{sketch}

Упрощение схемы -- техническая часть, придется учить $=($. Общая суть -- избавиться от $v_k$ и привести к виду, напоминающему метод тяжелого шарика и метод Чебышёва.

\section{Метод Ньютона}

\begin{theorem_ru}[О скорости сходимости метода Ньютона для задач оптимизации]
Пусть $f:\mathbb{R}^n\rightarrow \mathbb{R}$ дважды дифференцируема на $S=\left\{||x-x^*||\leq\frac{m}{\gamma M}\right\}$ 
при некотором $\gamma\geq 3/2$, $x^*$ -- точка минимума $f$ на $S$ и $mI\preceq\nabla^2 f(x^*)$ при $m>0$, для $\nabla^2 f$ на $S$ выполняется условие Липшица с константой $M$, $x_0\in S$, 
тогда для последовательности, генерируемой методом Нюьтона, выполняется
$$
||x_{k+1}-x^*||\leq \frac{M||x_k-x^*||^2}{2(m-M||x_k-x^*||)}
$$
\end{theorem_ru}

\begin{sketch}
С помощью формулы Ньютона-Лейбница показать, что
$$
x_k-x^*=[\nabla^2 f(x_k)]^{-1}G_k(x-x^*),
$$
показать, что $G_k=\mathcal{O}(x_k-x^*)$, а при достаточной близости $x_0$ к $x^*$ 
величина $||[\nabla^2 f(x_k)]^{-1}||$ ограничено.
\end{sketch}

\section{Субдифференциальное исчисление}
Свойства:
\begin{enumerate}
\item Замкнутость, выпуклость, ограниченность, непустота субдифференциала:\\
Замкнутость и выпуклость проверяется по определению. Для ограниченности и непустоты нужно рассмотреть эпиграф (надграфик).
\item В случае дифференцируемости градиент -- единственный субдифференциал, очевидно.
\item Субдифференциал сечения -- очевидно.
\item 
$$
f'(x;p)=\sup_{g\in \partial f(x)}g^Tp.
$$
Доказать, что $\phi(p)=f'(x;p)$ -- выпуклая функция и при этом $\partial \phi(0)=\partial f(x)$, равенство $\phi(p)=\phi(0)+\sup_{g\in\partial \phi(0)}g^Tp$ почти очевидно.
\item Линейность субдифференциала -- очевидно из предыдущего пункта.
\item Если $f_1, \ldots, f_m$ -- выпуклые функции, то для функции $f(x)=\max_{1\leq i\leq m}f_i(x)$ выполняется
$$
\partial f(x)=\conv\cup_{i\in I(x)}\partial f_i(x),
$$
где $I(x)=\{i~|~f_i(x)=f(x)\}$, $\conv X$ -- выпуклая оболочка множества $X$.\\
\begin{sketch}
Воспользоваться характеристикой через производную по направлению и двумя простыми фактами:
\begin{enumerate}
\item 
$$
f'(x;p)=\max_{i\in I(x)}f_i'(x;p)
$$
\item Для любых вещественных чисел $\beta_1, \ldots, \beta_n$ выполняется
$$
\max_{1\leq i\leq n}\beta_i=\max_{\alpha\in\Delta_k}\sum_{i=1}^n\alpha_i\beta_i
$$
\end{enumerate}
\end{sketch} 
\item Характеристика минимума -- тривиально.
\end{enumerate}

\subsection{Доказательство условий ККТ через субградиент}
\begin{lemma_ru}
Если $t^*=\min_{g(x)\leq 0_m}f(x)$, то
$$
\begin{cases}
f^*(t)\leq 0, & t\geq t^*,\\
f^*(t)\geq 0, & t\leq t^*.
\end{cases}
$$
\end{lemma_ru}
\begin{sketch}
Легко проверяется из определений $F$ и $f^*$.
\end{sketch}

Из леммы выводится
$$
x^*=\argmin_{g(x)\leq 0_m}f(x)\Leftrightarrow x^*=\argmin_x F(t^*, x).
$$
Применяя характеристику минимума к $F(t^*, x)$ получаем условия ККТ.

\section{Проективный градиентный спуск}
\subsection{Свойства проекции}
\begin{enumerate}
\item $(P_\mathcal{K}(x)-x)^T(y-P_\mathcal{K}(x))\geq 0~~y\in \mathcal{K}$.\\
Доказывается применением условий стационарности к проекции (проекция минимизирует расстояние).
\item $||P_\mathcal{K}(x)-P_\mathcal{•}mathcal{K}(y)||\leq ||x-y||$.\\
Доказывается применением первого пункта к парам $(x, P_\mathcal{K}(y))$, $(y, P_\mathcal{K}(x))$

\subsection{Анализ градиентного спуска}
Основной трюк заключается в том, чтобы получить неравенство вида
$$
||x_{k+1}-x^*||\leq ||x_k-x^*-\alpha(\nabla f(x_k)-\nabla f(x^*))||
$$
используя свойства проекции и точки $x^*$. Дальнейший анализ повторяет случай без проекции. 
\end{enumerate}



\end{document}