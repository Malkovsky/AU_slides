\documentclass[10pt]{beamer}
\usetheme{Madrid}
\usepackage[utf8]{inputenc}
\usepackage[russian]{babel}
\usepackage[OT1]{fontenc}
\usepackage{amsmath}
\usepackage{amsfonts}
\usepackage{amssymb}
\usepackage{graphicx}
\author{Мальковский Н.~В.}
\title[Введение]{Задачи оптимизации и поиска корней функции}
%\setbeamercovered{transparent} 
\setbeamertemplate{navigation symbols}{} 
%\logo{} 
\institute[СПбAУ]{Санкт-Петербургский Академический Университет}
\date{} 
%\subject{} 
\DeclareMathOperator*{\argmin}{argmin}
\DeclareMathOperator*{\argmax}{argmax}
\usecolortheme{beaver}


\newtheorem{theorem_ru}{Теорема}[]
\newtheorem{lemma_ru}{Лемма}[]
\newtheorem{corollary_ru}{Следствие}[]

\graphicspath{{image/}}
\newcommand{\Ima}{\text{Im}}

\begin{document}

\begin{frame}
\titlepage
\centering
\includegraphics[width=.23\textwidth]{logo.png}
\end{frame}

\begin{frame}{Задача нахождения корня функции}
Пусть $f:D\rightarrow \mathbb{R}^n$. \underline{Корнем} функции называется точка $x^*\in D$, такая что 
$$
f(x^*)=0_n.
$$
Традиционно задачу о нахождении корня функции попросту называют уравнением в случае $n=1$, и системой уравнений в случае $n>1$.
\end{frame}

\begin{frame}{Задача оптимизации}
Пусть $f:D\rightarrow \mathbb{R}^n$. \underline{Глобальным минимумом} функции $f$ на $D$ называется точка $x^*$, такая что
$$
\forall x\in D:~f(x)\geq f(x^*).
$$
\pause
Обозначается$$
\argmin_{x\in D} f(x).
$$
\pause
Аналогично, \textit{глобальный максимум} $f$ на $D$ -- такая точка $x^*$, что
$$
\forall x\in D:~f(x)\geq f(x^*),~x^*=\argmax_{x\in D} f(x).
$$
\pause
Задача оптимизации -- нахождение либо глобального минимума, либо глобального максимума.\\
\vspace{1em}
\pause
\textit{Замечание}. Задача максимизации $f$ -- это задача минимизации ``$-f$''. Большинство задач оптимизации формулируются в терминах нахождения минимума. Далее под $x^*$ будет обычно обозначаться точка глобального минимума.
\end{frame}

\begin{frame}{Связь задач нахождения минимума и корня функций}
Заметим, что
$$
f(x^*)=0~\Rightarrow~x^*=\argmin_x|f(x)|=\argmin_xf(x)^2.
$$
\pause
Таким нехитрым образом можно свести задачу нахождения корня к задаче нахождения минимума. В обратную сторону сведение можно получить только для дифференцируемых функций.
\end{frame}
\begin{frame}{Теорема Ферма}
\begin{theorem_ru}
$f:\mathcal{D}\subset \mathbb{R}^n$, $x^*$ -- внутренняя точка $\mathcal{D}$ и $f$ дифференцируема в точке $x^*$. Если $x^*$ -- точка минимума $f$ на $\mathcal{D}$, тогда выполняется условие:
$$
\nabla f(x^*)=0_n.
$$
\end{theorem_ru}
\pause
\textbf{Док-во}:
Так как $x^*$ -- внутренняя точка $\mathcal{D}$, то существует $\delta>0$ такое, что $\overline{B}(x^*, \delta)\in \mathcal{D}$. Если $\nabla f(x^*)\neq 0$, то для
$x=x^*-\frac{\delta}{2}\nabla f(x^*)$ выполняется $\nabla f(x^*)^T(x-x^*)=-\frac{\delta}{2}||\nabla f(x^*)||^2<0$, в силу дифференцируемости $f$ при достаточно малых $\alpha>0$ получаем
$$
f(x^*+\alpha (x-x^*))-f(x^*)=\alpha\nabla f(x^*)^T(x-x^*)+o(\alpha||x-x^*||)\leq
$$
$$
 \alpha \nabla f(x^*)^T(x-x^*)+\alpha ||x-x^*||\left(\frac{-\nabla f(x^*)^T(x-x^*)}{2||x-x^*||}\right)\leq \frac{\alpha}{2}\nabla f(x^*)^T(x-x^*)<0.
$$
\pause
Таким образом мы получили $f(x^*+\alpha(x-x^*))<f(x^*)$, для противоречия достаточно лишь гарантировать, что $x^*+\alpha (x-x^*)\in\overline{B}(x^*, \delta)$, что выполняется при $\alpha=\frac{1}{||\nabla f(x^*)||}$.~~ $\blacksquare$ 
\end{frame}

\begin{frame}{Стационарность}
Замечания:
\begin{itemize}[<+->]
\item Данную теорему обычно называют \underline{условием стационарности} или условиями оптимальности первого порядка. Точку, удовлетворяющей этим условием называют \underline{стационарной точкой} или критической точкой.
\item Условия стационарности не являются достаточными, как например для точки $0$ функции $f(x)=x^3$.
\item Точка $x^*$ называется \textit{локальным минимумом} $f$ на $D$, если существует $\epsilon>0$, что
$$
\forall x\in D,~||x-x^*||<\epsilon~f(x)\geq f(x^*).
$$
Если для для $f$ в точке $x^*$ гессиан положительно определен, то $x^*$ гарантированно является точка локального минимума. Это свойство принято называть \underline{условием оптимальности второго порядка}.
\end{itemize}
\end{frame}


\begin{frame}{Метод бисекции}
Пусть $f:[a,b]\rightarrow \mathbb{R}$, $f$ непрерывна и $f(a)<0<f(b)$. Требуется найти такую точку $x^*\in (a, b)$, что $f(x^*)=0$. Из непрерывности $f$ следует, что $x^*$ существует.
\begin{center}
\includegraphics[width=.7\textwidth]{bisection1}
\end{center}

\end{frame}

\begin{frame}{Метод бисекции}
Алгоритм:
\begin{itemize}
\item Разделить отрезок $[a,b]$ пополам и посмотреть значение $f$ посередине.
\item Если $f\left(\frac{a+b}{2}\right)>0$, отбросить правую половину
$$
b\leftarrow \frac{a+b}{2}.
$$
\item Если $f\left(\frac{a+b}{2}\right)<0$, отбросить правую половину
$$
a\leftarrow \frac{a+b}{2}.
$$
\item Если $f\left(\frac{a+b}{2}\right)=0$, то решение найдено.
\item Если $b-a>\epsilon$, повторить сначала.
\end{itemize}

\end{frame}

\begin{frame}{Метод бисекции}
\only<1-1>{
\begin{center}
\includegraphics[width=.8\textwidth]{bisection1}%
\end{center}
}
\only<2-2>{
\begin{center}
\includegraphics[width=.8\textwidth]{bisection2}%
\end{center}
}
\only<3-3>{
\begin{center}
\includegraphics[width=.8\textwidth]{bisection3}%
\end{center}
}
\only<4-4>{
\begin{center}
\includegraphics[width=.8\textwidth]{bisection4}%
\end{center}
}
\only<5-5>{
\begin{center}
\includegraphics[width=.8\textwidth]{bisection5}%
\end{center}
}
\end{frame}

\begin{frame}{Метод бисекции}
Свойства:
\begin{itemize}
\item На каждом этапе алгоритма выполняется свойство $f(a)<0<f(b)$, а значит алгоритм всегда работает с отрезком, который содержит корень.
\item Не находит точное значение корня, но может найти приближенное значение с любой заданной точностью.
\item Для нахождения корня с точностью $\epsilon>0$ необходимо сделать $\log_2\left(\frac{b-a}{\epsilon}\right)$ итераций.
\item * Иногда этот метод называют \textit{методом дихотомии}, в дискретном случае этом метод обычно называют \textit{двоичным} или \textit{бинарным поиском}.
\end{itemize}
\end{frame}

\begin{frame}{Тернарный поиск}
Пусть $f:[a,b]\rightarrow\mathbb{R}$, $\exists x^*\in (a, b):$ $f$ строго убывает на $[a, x^*]$, $f$ строго возрастает на $[x^*, b]$. Нужно найти минимум функции $f$ на $[a, b]$, т.е. точку $x^*$. Функцию, обладающую подобным свойством принято называть унимодальной.

\begin{center}
\includegraphics[width=.5\textwidth]{ternary1}
\end{center}

\end{frame}

\begin{frame}{Тернарный поиск}
Алгоритм:
\begin{itemize}
\item Выбрать любые две точки $a<m_1<m_2<b$.
\item Если $f(m_1)<f(m_2)$, то $x^*\notin [m_2, b]$,
$$
b\leftarrow m_2.
$$
\item Если $f(m_1)>f(m_2)$, то $x^*\notin [a, m_1]$.
$$
a\leftarrow m_1.
$$
\item Если $f(m_1)=f(m_2)$, то $x^*\notin  [m_2, b]\cup [a, m_1]$,
$$
a\leftarrow m_1,~b\leftarrow m_2.
$$
\item Если $b-a>\epsilon$, повторить сначала.
\end{itemize}
\end{frame}

\begin{frame}{Тернарный поиск}
\only<1-1>{
\begin{center}
\includegraphics[width=.65\textwidth]{ternary1}
\end{center}
}
\only<2-2>{
\begin{center}
\includegraphics[width=.65\textwidth]{ternary2}
\end{center}
}
\only<3-3>{
\begin{center}
\includegraphics[width=.65\textwidth]{ternary3}
\end{center}
}
\only<4-4>{
\begin{center}
\includegraphics[width=.65\textwidth]{ternary4}
\end{center}
}
\only<5-5>{
\begin{center}
\includegraphics[width=.65\textwidth]{ternary5}
\end{center}
}
\end{frame}

\begin{frame}{Тернарный поиск}
Свойства:
\begin{itemize}
\item На каждом этапе алгоритма выполняется свойство $a<x^*<b$.
\item Опять же, не получится найти минимум точно, только с любой заданной точностью.
\item Если точки делят отрезок на $3$ равные части, то для нахождения минимума с точностью $\epsilon>0$ необходимо сделать $\log_{\frac{3}{2}}\frac{b-a}{\epsilon}$ итераций.
\end{itemize}
\end{frame}


\end{document}
